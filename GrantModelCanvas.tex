\documentclass{article}

% adapted "business" version from latex code:
% https://rememberthecmd.blogspot.com/2015/02/draw-business-model-generation-canvas.html
% and the "grant" version (not latex):
% https://gregglab.neuro.utah.edu/2018/10/30/the-grant-model-canvas-for-developing-great-grants/

\usepackage[landscape,margin=0in]{geometry}
%\usepackage[english]{babel}
\usepackage[utf8]{inputenc}
\pagenumbering{gobble} % supress page numbers
\usepackage{hyperref}
\usepackage{datetime2} % YYYY-MM-DD format!
% Sans-serif fonts
\renewcommand{\rmdefault}{phv}
\renewcommand{\sfdefault}{phv} 

\usepackage{tikz}

\title{\vspace{-1em}The Grant Model Canvas}
\author{Alejandro Ochoa --- \href{mailto:alejandro.ochoa@duke.edu}{\nolinkurl{alejandro.ochoa@duke.edu} } --- \url{https://ochoalab.github.io/}}
\date{
  \textbf{\color{black!50}Version:} \today
  ~\textbf{\color{black!50}---}
  \textbf{\color{black!50}Due date:} 2019-06-05
}


\begin{document}
\maketitle

\vspace{-1em}
\centering
\def\layersep{9.7em}
\def\layerwidth{75em}

\makebox[\textwidth][c]{
  \begin{tikzpicture}[
      % Define block parameters (mostly shape)
      bloc/.style={
        rectangle, rounded corners,
        draw=black!30, very thick, inner sep=0,
      },
      bloc1/.style={
        bloc,
        text width = \layerwidth/5*0.95,
        minimum width = \layerwidth/5,
        minimum height= 4*\layersep
      },
      bloc2/.style={
        bloc,
        text width = \layerwidth/5*0.95,
        minimum width=\layerwidth/5,
        minimum height=2*\layersep
      },
      bloc3/.style={
        bloc,
        text width=\layerwidth/2*0.95,
        minimum width=\layerwidth/2,
        minimum height=\layersep
      },
      title/.style={
        anchor=north west,
        color=black!50,
        font=\bfseries
      },
    ]
    
    %%%%%%%%%%%%%%%%%%%%%%%%
    %%% DRAW THE CANVAS
    %%%%%%%%%%%%%%%%%%%%%%%%

    % The 7 top blocks

    % We start drawing the top left bloc here
    % 1st, the block, then the title, then the post-it
    \node[bloc1] (b0) at (0*\layerwidth/10,4*\layersep) {
      \begin{itemize}
      \item Text
      \item More text
      \end{itemize}
    };
    \node[title] at (b0.north west) {\underline{1. Problem}};

    \node[bloc2] (b1) at (2*\layerwidth/10,5*\layersep) {
      \begin{itemize}
      \item Text
      \item More text
      \end{itemize}
    };
    \node[title] at (b1.north west) {\underline{2. Target audience}};

    \node[bloc2] (b2) at (2*\layerwidth/10,3*\layersep) {
      \begin{itemize}
      \item Text
      \item More text
      \end{itemize}
    };
    \node[title] at (b2.north west) {\underline{3. Solution}};

    \node[bloc1] (b3) at (4*\layerwidth/10,4*\layersep) {
      \begin{itemize}
      \item Text
      \item More text
      \end{itemize}
    };
    \node[title] at (b3.north west) {\underline{6. Value Propositions}};

    \node[bloc2] (b4) at (6*\layerwidth/10,5*\layersep) {
      \begin{itemize}
      \item Text
      \item More text
      \end{itemize}
    };
    \node[title] at (b4.north west) {\underline{7. Unfair Advantage}};

    \node[bloc2] (b5) at (6*\layerwidth/10,3*\layersep) {
      \begin{itemize}
      \item Text
      \item More text
      \end{itemize}
    };
    \node[title] at (b5.north west) {\underline{8. Expected Impact}};

    \node[bloc2] (b6) at (8*\layerwidth/10,5*\layersep) {
      \begin{itemize}
      \item Text
      \item More text
      \end{itemize}
    };
    \node[title] at (b6.north west) {\underline{9. Key Resources Needed}};

    \node[bloc2] (b7) at (8*\layerwidth/10,3*\layersep) {
      \begin{itemize}
      \item Text
      \item More text
      \end{itemize}
    };
    \node[title] at (b7.north west) {\underline{10. Novelty}};

    %%%% The 2 bottom blocks
    \node[bloc3] (b8) at (1.5*\layerwidth/10,1.5*\layersep) {
      \begin{itemize}
      \item Text
      \item More text
      \end{itemize}
    };
    \node[title] at (b8.north west) {\underline{4. What is Known?}};

    \node[bloc3] (b9) at (6.5*\layerwidth/10,1.5*\layersep) {
      \begin{itemize}
      \item Text
      \item More text
      \end{itemize}
    };
    \node[title] at (b9.north west) {\underline{5. What is Unknown?}};


  \end{tikzpicture}
}

\end{document}
